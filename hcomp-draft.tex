%File: formatting-instruction.tex
\documentclass[letterpaper]{article}
\usepackage{aaai}
\usepackage{times}
\usepackage{helvet}
\usepackage{courier}
\frenchspacing
\pdfinfo{
/Title (Formatting Instructions for Authors Using LaTeX)
/Subject (AAAI Publications)
/Author (AAAI Press)}
\setcounter{secnumdepth}{0}  
 \begin{document}
% The file aaai.sty is the style file for AAAI Press 
% proceedings, working notes, and technical reports.
%
\title{Exploring Repeatability on Mechanical Turk}
\author{Kristal Curtis\\
UC Berkeley\\
465 Soda Hall\\
Berkeley, California 94720\\
}
\maketitle
\begin{abstract}
Repeatability is very desirable.
\end{abstract}

\section{Introduction}

Crowdsourcing has provided numerous opportunities for people in academia and industry to access a numerous pool of workers willing to perform useful work for modest compensation.  Many people have taken advantage of crowdsourcing for achieving tasks that are still difficult for computers to complete yet are quite simple for humans, such as image labeling and object characterization.  Recently, myriad researchers from computer science as well as disciplines in the social sciences like sociology and economics have turned to Mechanical Turk (MTurk), a popular and flexible crowdsourcing platform, for running human subject experiments.

Due to its large size and impersonal nature, Mechanical Turk gives the illusion of uniformity.  However, the reality is that it is incredibly heterogeneous, with participants from around the globe \cite{Ipeirotis:2010a}.  In addition, some users of Mechanical Turk (\textit{requesters}, in MTurk parlance) have noted that some workers (ie, \textit{Turkers}) are more active than others \cite{Franklin:2011}.

A concern that has been raised by both employers and experimenters is the issue of \textit{repeatability}; that is, if a batch of tasks is run under different conditions (eg, different time of day, different day of week), the results may vary.  In some cases, this may be a concern because the answer(s) provided by the Turkers may be different, in terms of actual value(s) and/or overall quality.   In others, the response time (either time to first answer or time to batch completion) may also be  crucial yet highly variable.

In this work, we explore various factors that serve as obstacles to repeatability.  We also offer some ideas about how to improve repeatability.

\section{Obstacles to Repeatability}

In this section, we investigate the impact of several factors that may serve as obstacles to repeatability.

\subsection{Zipfian Turker Pool}
Some recent studies have shown that for a given group of tasks (ie, a HIT group, where each task is a HIT, or Human Intelligence Task), a small number of Turkers complete a disproportionate amount of the work offered \cite{Franklin:2011, Heer:2010}.  In this work, we will refer to these overly-active Turkers as \textit{super Turkers} (\cite{Heer:2010} refers to them as streakers).

Let us refer to a given HIT group as \(G\).  The members of the set \(T_S(G_i)\) are the super Turkers who completed HITs for the \(i\)th execution of \(G\), where we assume that \(G\) is executed \(n\) times (ie, at \(n\) different occasions).  For our purposes, a Turker \(t \in T_S(G_i)\) if he/she completes at least \(k\) HITs, where \(H = |G|\) and \(0 < k \le H\).  We will explain how to select \(k\) later on.  % When?

First, we would like to determine the nature of the intersection between \(T_S(G_i)\) and \(T_S(G_j)\), where \(i \ne j\).  It will likely be impacted by the similarity between occasions \(i\) and \(j\); ie, if \(i\) and \(j\) are very different times or dates, you would expect \(T_S(G_i) \cap T_S(G_j) = \emptyset\).

Our hypothesis is that for executions \(G_i\) and \(G_j\), \(i \ne j\), \(T_S(G_i) \cap T_S(G_j)\) is small wrt both \(|T_S(G_i)|\) and \(|T_S(G_j)|\), and that this will cause the results of \(G_i\) and \(G_j\) to be different.  % need to be more precise re: "small"

To validate our hypothesis, we propose the following meta-experiment:
\begin{itemize}
\item Given:  experiment \(E\), number of occasions \(n\)
\item Obtain \(G_1(E), \dots, G_n(E)\)
\item Obtain \(T_S(G_1), \dots, T_S(G_n)\)
\item Determine the intersections among super Turker sets \(T_S(G_i), i \in \{1, \dots, n\} \).
\item Analyze the impact of \(T_S(G_i)\) on the results of \(G_i\).
\end{itemize}

We will explore these ideas in the context of a concrete experiment in the Experiments section.

\subsection{Contention with Other Tasks}

Another factor that could impact repeatability of a HIT group is the \textit{task context}; ie, the number and types of other tasks that are currently live on the MTurk platform.  For example, \cite{Franklin:2011} observed that even one's own tasks could compete with each other.  Therefore, we will also investigate the impact of task context on repeatability.  Task context seems more likely to affect response time than actual result values.

In order to measure the number of active HITs on MTurk during the execution of task group \(G_i\), we will scrape the MTurk website once every ten minutes and report the average value observed during the lifetime of \(G_i\), which is defined as the time \(G_i\) is posted to the time when all of \(G_i\)'s tasks have been completed (ie, 100\% of each HIT's assignments).

\subsection{External Events}

We also expect that the incidence of external events such as holidays and natural disasters will impact repeatability.  In \cite{Heer:2010}, for example, the authors note that they had to correct their HIT results that were obtained on a holiday so that they would be comparable with the rest of their results.  

Avoiding posting HITs on holidays seems like an obvious workaround; however, the global nature of the MTurk workforce complicates this attempt since Turkers may observe holidays of which the requesters, who are currently all based in the US due to platform restrictions, are unaware.  

Unpredictable events that could impact repeatability include natural disasters and unreliable infrastructure, which often disproportionately affect the developing world.  Since many Turkers are in India, where they can receive payment in their own currency, this is likely to be an issue.  

At this time, we do not attempt to address these issues.  However, we do recommend that requesters keep these in mind as potential causes of anomalous results.

\section{Experiments}

% explain my experiment E:  New Yorker experiment

% first question:  are the results different?
% Don't forget to correct results using random to un-shuffle (get code from amp poster prep)

% 2nd question:  look at the super turker sets.  are they different?
% wrt results (% of vote received by each caption overall), response time (show cdf)

% 3rd question:  look at task context.  impact on results?  impact on response time?

% if possible, do this for another experiment too -- clothing categorization

\section{Conclusion}

\bibliographystyle{acm}
\bibliography{/Users/kcurtis/Desktop/Readings.bib}

\end{document}
